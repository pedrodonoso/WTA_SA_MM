\section{Introducci\'on}
% Una explicaci\'on breve del contenido del informe, es decir, detalla: Prop\'osito, Estructura del Documento, Descripci\'on (muy breve) del Problema y Motivaci\'on.

\subsection{Propósito}
Este informe fue desarrollado con el propósito de que lector conozca que existen diferentes forma de abordar un problema de optimización combinatoria, específicamente el problema de asignación de armas a objetivos de guerra, para que así pueda aplicar este conocimiento en problemas similares.

\subsection{Descripción del problema}

% En esta sección no hay citas/referencias a otros artículos que respalden la información presentada.

% No se mencionan problemas relacionados, como por ejemplo el assignment problem, que es el problema general de asignación en la literatura.

En este informe se dispondrá al lector un problema relacionado con las estrategias de guerra, específicamente el problema de Asignación de objetivos de armas, este problema conocido como \textit{Weapon Target Assignment} (WTA) consiste en la asignación de armas amigas a los objetivos hostiles con el fin de proteger los activos amigos o destruir los objetivos hostiles, es considerado un problema NP-completo, por consiguiente tiene variaciones en su formulación y diversas formas de obtener una solución.
Este documento busca expresarle al lector la esencia del problema para luego estudiar las diversas formas de encontrar soluciones incompletas, heurísticas a utilizar y compararlas, comprendiendo así que no sólo existe una forma para encontrar soluciones.

\subsection{Estructura del documento}
% No se presenta la estructura del documento.
Primero se presentará la Definición del problema, donde se explicará el caso general del problema de una manera verbal para que sea entendido fácilmente y se tenga una idea general de este informe. En segundo lugar se tendrá el Estado del Arte, donde se mostrarán los estudios realizados con antelación, con la finalidad de proveer al lector información relevante que le sirva para ver las diferentes formas de abordar el problema de optimización combinatoria.
En tercer lugar estarán los Modelos Matemáticos tanto para el caso general del WTA como para un caso específico de este, es necesario tener presenta las variables, restricciones y funciones objetivo para poder utilizar los algoritmos que se presentarán en el documento. Por último se tendrán las Conclusiones, que comparará los métodos planteados en el informe y se dará un pequeño resumen del más prometedor. 

\subsection{Motivación}
La motivación para crear este informe es aprender como un problema puede ser resuelto de diferentes formas, con el fin de posteriormente intentar crear una solución al WTA recurriendo al conocimiento extraído de este informe, ya sea utilizando una de esas técnicas o combinándolas.
